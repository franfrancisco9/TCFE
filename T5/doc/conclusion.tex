\newpage
\section{Conclusion}
\label{sec:conclusion}

In this laboratory we dimensioned and implemented a BandPass Filter (BPF) using an OP-AMP.

We can now compare the obtained results by theory and by simulation:

\begin{table}[h]
    \centering
    \begin{tabular}{|c|c|c|}
    \hline
    {\bf Parameter} & {\bf Theoretical Value}& {\bf Simulation Value}\\
    \hline\hline
      Low Frequency [Hz] &  & \\
    \hline
    High Frequency [Hz] &  & \\
    \hline
   Central Frequency [Hz] &  & \\
   \hline
     $Z{input}$ &  & \\
    \hline
     $Z{output}$ &  & \\
    \hline
      Gain [db] &  & \\
    \hline
    \end{tabular}
    \caption{Comparison of the theoretical and simulation values.}
    \label{tab:values}
\end{table}

Contrary to our goal for the laboratory, there are some clear differences between the results. This could be due to the model applied in Ngspice being much more realistic as well as its parameters being more complex than the one analysed in the theoretical part or the fact that the components of the circuit, specially the OP-AMP, are not linear. 
Despite all of this, we are satisfied with our results and we consider the model used to be valid.


Final merit obtained was ?? (Ngspice)

